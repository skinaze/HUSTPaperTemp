%本文设置图标样式

%设置图标标题的计数方式
\makeatletter
\renewcommand{\thefigure}{\ifnum \c@section>\z@ \thesection-\fi \@arabic\c@figure}
\renewcommand{\thetable}{\ifnum \c@section>\z@ \thesection-\fi \@arabic\c@table}
\makeatother

%设置标题的样式
\usepackage{caption}
\captionsetup{font={normalsize,bf,singlespacing}}
\captionsetup[figure]{position=bottom,skip={0pt}}
\captionsetup[table]{position=top,skip={0pt}}
\setlength{\textfloatsep}{0pt}
\setlength{\floatsep}{0pt}
\setlength{\intextsep}{6pt}
\setlength{\abovecaptionskip}{0pt}
\setlength{\belowcaptionskip}{0pt}

%重新设置图表autoref
\newcommand{\figureautorefname}{图}
\newcommand{\tableautorefname}{表}

%使用tabular库并定义新的左右中格式
\usepackage{tabularx}
\newcolumntype{L}{X}
\newcolumntype{C}{>{\centering \arraybackslash}X}
\newcolumntype{R}{>{\raggedright \arraybackslash}X}

%添加复杂的表格需求库
\usepackage{multirow}
\usepackage{makecell}
\usepackage{diagbox}
\usepackage{booktabs}
%设置三线表格式
\setlength{\heavyrulewidth}{1.5pt}
\setlength{\lightrulewidth}{0.5pt}
\setlength{\cmidrulewidth}{0.5pt}
\setlength{\aboverulesep}{0pt}
\setlength{\belowrulesep}{0pt}
\setlength{\abovetopsep}{0pt}
\setlength{\belowbottomsep}{0pt}

%添加图需要的库
\usepackage{graphicx}

%重新定义图和表的浮动环境,使其方便使用
\newenvironment{generalfig}[2]{
	\def \figcaption {#1}
	\def \figlabel {#2}
	\begin{figure}[htbp]
		\centering
}{
		\vspace{-2em} %缩减莫名其妙的空行,暂不明问题缘由
		\caption{\figcaption} \label{\figlabel}
	\end{figure}
}
\newenvironment{generaltab}[2]{
	\def \tabcaption {#1}
	\def \tablabel {#2}
	\begin{table}[htbp]
		\caption{\tabcaption} \label{\tablabel}
		\zihao{5}
		\centering
}{
	\end{table}
}
