%请使用XeLaTeX进行编译
\documentclass[a4paper,cs4size,UTF8,winfonts,boldfont,slantfont]{ctexart}
	
	%应用文档设置
	%本文用于设置页边距
\usepackage{geometry}
\geometry{left=3.18cm,right=3.18cm,top=2.54cm,bottom=2.54cm}
\geometry{headsep=2em,footskip=2em}
	%本文设置字体

%用于设置非常用字体
\usepackage{fontspec}

%论文封面使用华文中宋
\setCJKfamilyfont{hwzs}{STZhongsong}
\newcommand{\huawenzhongsong}{\CJKfamily{hwzs}}

	%本文用于生成页眉页脚

%用于绘制页脚横线
\usepackage{tikz}

%使用fancyhdr库
\usepackage{fancyhdr}
\pagestyle{fancy}
\fancyhf{} %清空原有样式
%设置页眉样式
\fancyhead[C]{\huawenzhongsong \zihao{5} 华 \space 中 \space 科 \space 技 \space 大 \space 学 \space 毕 \space 业 \space 设 \space 计(论  \space 文)}
%设置页脚样式
\fancyfoot[C]{\huawenzhongsong \zihao{5} \thepage}
\fancyfoot[L]{\tikz{\node (A) at (0,0) {}; \draw (0,1pt)--(6.7cm,1pt);}}
\fancyfoot[R]{\tikz{\node (A) at (0,0) {}; \draw (0,1pt)--(-6.7cm,1pt);}}

%修改原plain格式
\fancypagestyle{plain}{%
	\fancyhf{} %清空原有样式
	%设置页眉样式
	\fancyhead[C]{\huawenzhongsong \zihao{5} 华 \space 中 \space 科 \space 技 \space 大 \space 学 \space 毕 \space 业 \space 设 \space 计(论  \space 文)}
	%设置页脚样式
	\fancyfoot[C]{\huawenzhongsong \zihao{5} \thepage}
	\fancyfoot[L]{\tikz{\node (A) at (0,0) {}; \draw (0,1pt)--(6.7cm,1pt);}}
	\fancyfoot[R]{\tikz{\node (A) at (0,0) {}; \draw (0,1pt)--(-6.7cm,1pt);}}
}
	%重新设置目录
\usepackage{tocloft}
\usepackage{hyperref}
\hypersetup{hidelinks} %去掉目录红框
\renewcommand{\cfttoctitlefont}{\hfill \heiti \zihao{-2} \bfseries}
\renewcommand{\contentsname}{目\hspace{2em}录}
\renewcommand{\cftaftertoctitle}{\hfill}
\renewcommand{\cftbeforeloftitleskip}{0.5em}
\renewcommand{\cftafterloftitleskip}{0.5em}
\renewcommand{\cftsecdotsep}{\cftdotsep} %设置Section引导点
\renewcommand{\cftbeforesecskip}{0em} %设置段间距
\renewcommand{\cftpartfont}{\songti \bfseries}%设置Part字体
\renewcommand{\cftsecfont}{\songti \bfseries}%设置Section字体

%使用条件语句
\usepackage{xstring}

%定义两个新语句
%第一个语句更改fancyhdr样式
\newcommand{\plainfooterstyle}[1]{
	\IfStrEqCase{#1}{{nopagenum}{}
		{pagenum}{\footstyle}
		{pagenumtoc}{\footstyle}}
}
%第二个语句在目录中添加目录标签
\newcommand{\addtoctotoc}[1]{
	\IfStrEqCase{#1}{{nopagenum}{}
		{pagenum}{}
		{pagenumtoc}{\phantomsection
			\addcontentsline{toc}{section}{目录}}}
}

%设置新的生成目录命令
\let \oldtableofcontens \tableofcontents
\newcommand{\maketoc}[1][nopagenum]{
	%修改hdr原plain格式
	\fancypagestyle{plain}{%
		\fancyhf{} %清空原有样式
		\headstyle
		\plainfooterstyle{#1}
	}
	%设置目录hdr和页码选项
	\clearpage
	\pagestyle{plain}
	\addtoctotoc{#1}
	\tableofcontents
	\clearpage
	\pagestyle{main}
}
	%本文设置段落样式

%设置1.5倍行距
\renewcommand{\baselinestretch}{1.62}

%设置各个标题样式
%不需要使用part和chapter层级

\CTEXsetup[format={\clearpage \centering \heiti \bfseries \zihao{-2}}]{section} %节
\CTEXsetup[format={\raggedright \heiti \bfseries \zihao{4}}]{subsection} %小节
\CTEXsetup[format={\raggedright \heiti \bfseries \zihao{-4}}]{subsubsection} %小小节
\CTEXsetup[beforeskip={0em}]{paragraph}
\CTEXsetup[beforeskip={0em}]{subparagraph}
	%本文设置图标样式

%设置图标标题的计数方式
\makeatletter
\renewcommand{\thefigure}{\ifnum \c@section>\z@ \thesection-\fi \@arabic\c@figure}
\renewcommand{\thetable}{\ifnum \c@section>\z@ \thesection-\fi \@arabic\c@table}
\makeatother

%t添加复杂的表格需求库
\usepackage{multirow}
\usepackage{makecell}
\usepackage{diagbox}
\usepackage{booktabs}
	%本文设置两个新环境,分别是中英文摘要环境

%中文摘要环境
\newenvironment{cnabstract}[1]{
	\newcommand{\cnkeyword}{#1}
	\clearpage 
	\addcontentsline{toc}{section}{摘要}
	\begin{center} \heiti \bfseries \zihao{-2} 摘 \hspace{2em} 要 \end{center}
	
}{
	\vspace{1em}
	\paragraph{\heiti \bfseries 关键词:} \cnkeyword
}

%英文摘要环境
\newenvironment{enabstract}[1]{
	\newcommand{\enkeyword}{#1}
	\clearpage 
	\addcontentsline{toc}{section}{Abstract}
	\begin{center} \bfseries \zihao{-2} Abstract \end{center}
	
}{
	\vspace{1em}
	\paragraph{\bfseries Key Words: } \enkeyword
}
	%应用默认页面
	%本文重置maketitle

%添加几个新的文档属性
\def \defschool {}
\def \defclassnum {}
\def \defstunum {}
\def \definstructor {}
\newcommand{\school}[1]{\def \defschool {#1}}
\newcommand{\classnum}[1]{\def \defclassnum {#1}}
\newcommand{\stunum}[1]{\def \defstunum {#1}}
\newcommand{\instructor}[1]{\def \definstructor {#1}}

%重置命令maketitle
\makeatletter
\renewcommand{\maketitle}[1][12em]{
	\begin{titlepage}
		\begin{center}
			\vspace*{4em}
			\includegraphics[height=1.61cm]{HUSTGreen.png}\\
			\vspace*{2em}
			{\zihao{-0} \huawenzhongsong \bfseries 本科生毕业设计[论文]}\\
			\vspace{6em}
			
			{\zihao{2} \heiti \bfseries \@title}\\
			\vspace{6em}
			{\zihao{3} \huawenzhongsong 
				\renewcommand\arraystretch{1.5}
				\begin{tabular}{lc}
					\makebox[4em][s]{院 \hfill 系} &
					\underline{\makebox[#1]{\defschool}} \\
					\makebox[4em][s]{专业班级} &
					\underline{\makebox[#1]{\defclassnum}} \\
					\makebox[4em][s]{姓 \hfill 名} &
					\underline{\makebox[#1]{\@author}} \\
					\makebox[4em][s]{学 \hfill 号} &
					\underline{\makebox[#1]{\defstunum}} \\
					\makebox[4em][s]{指导教师} &
					\underline{\makebox[#1]{\definstructor}} \\
			\end{tabular}}\\
			\vspace{4em}
			{\zihao{3} \huawenzhongsong \@date}\\
		\end{center}
	\end{titlepage}
}
\makeatother
	%本文新加命令makestatement生成声明部分

%使用条件语句
\usepackage{xstring}
%使用特殊符号
\usepackage{amssymb}

%定义三个条件语句
\newcommand{\encryption}[1]{
	\IfStrEqCase{#1}{{empty}{$\square$}
		{true}{$\text{\rlap{\checkmark}}\square$ }
		{false}{$\square$}}
	[$\square$]
}
\newcommand{\nonencryption}[1]{
	\IfStrEqCase{#1}{{empty}{$\square$}
		{true}{$\square$}
		{false}{$\text{\rlap{\checkmark}}\square$}}
	[$\square$]
}
\newcommand{\encryptionyear}[2]{
	\IfStrEqCase{#1}{{empty}{}
		{true}{#2}
		{false}{}}
}

%设置命令
\newcommand{\makestatement}[2][0]{
	\clearpage
	\thispagestyle{empty}
	\vspace*{1em}
	\begin{center}
		\heiti \zihao{-2} \bfseries
		学位论文原创性声明
	\end{center}
	
	本人郑重声明:所呈交的论文是本人在导师的指导下独立进行研究所取得的研究成果。除了文中特别加以标注引用的内容外,本论文不包括任何其他个人或集体已经发表或撰写的成果作品。本人完全意识到本声明的法律后果由本人承担。
	
	\rightline{作者签名:\hspace{6em} 年 \hspace{2em} 月 \hspace{2em} 日}
	\vspace{4em}
	
	\begin{center}
		\heiti \zihao{-2} \bfseries
		学位论文版权使用授权书
	\end{center}

	本学位论文作者完全了解学校有关保障、使用学位论文的规定,同意学校保留并向有关学位论文管理部门或机构送交论文的复印件和电子版,允许论文被查阅和借阅。本人授权省级优秀学士论文评选机构将本学位论文的全部或部分内容编入有关数据进行检索,可以采用影印、缩印或扫描等复制手段保存和汇编本学位论文。
	
	\begin{tabbing}
		\hspace{2em}本学位论文属于 \= 1、保密\hspace{1em} \=\encryption{#2}
		,在\makebox[2em][c]{\encryptionyear{#2}{#1}}年解密后适用本授权书。\\
		\>2、不保密\>\nonencryption{#2}
		。\\
		\>(请在以上相应方框内打“$\checkmark$”)
	\end{tabbing}
	
	\rightline{作者签名:\hspace{6em} 年 \hspace{2em} 月 \hspace{2em} 日}
	
	\rightline{导师签名:\hspace{6em} 年 \hspace{2em} 月 \hspace{2em} 日}
	
	\newpage
}
	
	%进行个人信息设置
	\title{论文题目} %论文题目
	\author{作者姓名} %作者姓名
	\date{\today} %日期,默认当日
	\school{院系名称} %院系名称
	\classnum{专业班级} %专业班级
	\stunum {U201300000} %学号
	\instructor{指导教师姓名} %指导教师姓名
	
\begin{document}
	%生成标题页
	%如果要改变填写横杠长度,请这样使用:\maketitle[15em]
	\maketitle
	
	%生成声明与授权书页
	%使用方法:\makestatement[保密年数]{empty/true/false}
	%其中:empty为不填;true为保密,请填写保密年数;false为不保密
	\makestatement{empty}
	
	%摘要页码为大写罗马数字
	\pagenumbering{Roman}
	%填写中文摘要内容和关键字
	\begin{cnabstract}{关键词1;关键词2;关键词3}
		在正文中添加空行可以实现换行功能
		
		摘要内容摘要内容摘要内容摘要内容摘要内容摘要内容摘要内容摘要内容摘要内容摘要内容摘要内容摘要内容摘要内容摘要内容摘要内容摘要内容摘要内容摘要内容摘要内容摘要内容摘要内容摘要内容摘要内容摘要内容摘要内容
		
		摘要内容摘要内容摘要内容摘要内容摘要内容摘要内容摘要内容摘要内容摘要内容摘要内容摘要内容摘要内容摘要内容摘要内容摘要内容摘要内容摘要内容摘要内容摘要内容摘要内容摘要内容摘要内容摘要内容摘要内容摘要内容
	\end{cnabstract}
	%填写英文摘要内容和关键字
	\begin{enabstract}{Key1; Key2; Key3}
		This is abstract. This is abstract. This is abstract. This is abstract. This is abstract. This is abstract. This is abstract. This is abstract. This is abstract. This is abstract. This is abstract. This is abstract. 
		
		This is abstract. This is abstract. This is abstract. This is abstract. This is abstract. This is abstract. This is abstract. This is abstract. This is abstract. This is abstract. This is abstract. This is abstract. 
	\end{enabstract}
	
	%生成目录
	\clearpage
	\pagenumbering{arabic} %正文页码为阿拉伯数字
	\thispagestyle{fancy}
	\phantomsection
	\addcontentsline{toc}{section}{目录}
	\tableofcontents
	
	%正文内容从这里开始
	\section{第一节 The first Section}
	这是小四号的正文字体,段间距1.5倍
	
	通过空一行实现段落换行,仅仅是回车并不会产生新的段落
	\subsection{第一小节}
	\subsubsection{第一小小节}
	\subsubsection{第二小小节}
	\paragraph{段落}段落具体文章在此段落具体文章在此段落具体文章在此段落具体文章在此段落具体文章在此段落具体文章在此段落具体文章在此段落具体文章在此段落具体文章在此
	\subparagraph{小段落}段落具体文章在此段落具体文章在此段落具体文章在此段落具体文章在此段落具体文章在此段落具体文章在此段落具体文章在此段落具体文章在此
	\subsection{第二小节}
	这是一大段文字这是一大段文字这是一大段文字这是一大段文字这是一大段文字这是一大段文字这是一大段文字这是一大段文字这是一大段文字这是一大段文字这是一大段文字这是一大段文字这是一大段文字这是一大段文字这是一大段文字这是一大段文字这是一大段文字这是一大段文字这是一大段文字这是一大段文字这是一大段文字这是一大段文字这是一大段文字这是一大段文字这是一大段文字这是一大段文字这是一大段文字这是一大段文字这是一大段文字这是一大段文字这是一大段文字这是一大段文字这是一大段文字这是一大段文字这是一大段文字这是一大段文字这是一大段文字这是一大段文字这是一大段文字这是一大段文字这是一大段文字这是一大段文字这是一大段文字这是一大段文字这是一大段文字这是一大段文字这是一大段文字这是一大段文字这是一大段文字这是一大段文字这是一大段文字这是一大段文字这是一大段文字这是一大段文字这是一大段文字这是一大段文字这是一大段文字这是一大段文字这是一大段文字这是一大段文字这是一大段文字这是一大段文字这是一大段文字这是一大段文字这是一大段文字这是一大段文字这是一大段文字这是一大段文字这是一大段文字这是一大段文字这是一大段文字这是一大段文字这是一大段文字这是一大段文字这是一大段文字这是一大段文字这是一大段文字这是一大段文字这是一大段文字这是一大段文字这是一大段文字这是一大段文字这是一大段文字这是一大段文字这是一大段文字这是一大段文字这是一大段文字这是一大段文字这是一大段文字这是一大段文字这是一大段文字这是一大段文字这是一大段文字这是一大段文字这是一大段文字这是一大段文字这是一大段文字这是一大段文字这是一大段文字这是一大段文字这是一大段文字这是一大段文字这是一大段文字这是一大段文字这是一大段文字这是一大段文字这是一大段文字这是一大段文字这是一大段文字这是一大段文字这是一大段文字这是一大段文字这是一大段文字这是一大段文字这是一大段文字这是一大段文字这是一大段文字这是一大段文字这是一大段文字这是一大段文字这是一大段文字这是一大段文字这是一大段文字这是一大段文字这是一大段文字这是一大段文字这是一大段文字这是一大段文字这是一大段文字这是一大段文字这是一大段文字这是一大段文字这是一大段文字这是一大段文字这是一大段文字这是一大段文字这是一大段文字这是一大段文字这是一大段文字
	\section{第二节}
		新的大节会自动出现在新的一页上
		
		请注意目录和参考文献的编译过程
	\section{第三节}
	这是一个引用的范例\cite{Stone_1998}
	
	这样可以添加一个不标注的引用\nocite{9787508342894}
	
	这样可以添加所有bib文件中的参考文献\nocite{*}
	
	\section{测试多页目录的效果}
	wtf
	\subsection{测试多页目录的效果}
	\subsection{测试多页目录的效果}
	\subsection{测试多页目录的效果}
	\subsection{测试多页目录的效果}
	\subsection{测试多页目录的效果}
	\subsection{测试多页目录的效果}
	\subsection{测试多页目录的效果}
	abc
	\section{测试多页目录的效果}
	\subsection{测试多页目录的效果}
	\subsection{测试多页目录的效果}
	\subsection{测试多页目录的效果}
	\subsection{测试多页目录的效果}
	\subsection{测试多页目录的效果}
	\subsection{测试多页目录的效果}
	\subsection{测试多页目录的效果}
	\subsection{测试多页目录的效果}
	whatever
	
	%生成参考文献
	%添加到目录
	\clearpage
	\phantomsection
	\addcontentsline{toc}{section}{参考文献}
	%使用方法:\bibliography{参考文件1文件名, 参考文献2文件名, ...}
	\bibliography{Bibs/mybib}
	
	\begin{appendices}
		\section{这是第一个附录}
		这里是附录环境,其中的section、subsection、subsubsection已经变为附录的样式,并且会以这种样式加入目录中
		\subsection{附录可以有小节}
		\subsubsection{附录中也可以有小小节}
	\end{appendices}
	
\end{document}